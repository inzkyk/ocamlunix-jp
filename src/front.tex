%------------------------------------------------------------------------------
% Copyright (c) 1991-2014, Xavier Leroy and Didier Remy.
%
% All rights reserved. Distributed under a creative commons
% attribution-non-commercial-share alike 2.0 France license.
% http://creativecommons.org/licenses/by-nc-sa/2.0/fr/
%
% Translation by Daniel C. Buenzli
% 英語版の日本語への翻訳: Yuki (github: inzkyk)
% ------------------------------------------------------------------------------

\maketitle
\newpage

%% Copyright page
\begin{copyrightnotice}
\textcopyright{} 1991, 1992, 2003, 2004, 2005, 2006, 2008, 2009, 2010 \\
\myauthors, \textsc{inria} Rocquencourt.\\
Rights reserved.
\ifhtmlelse
    {Consult the \href{LICENSE}{license.}  \href{\licenseURL}%
      {\imgsrc[alt="CreativeCommons License" class="ccimage"]%
        {http://i.creativecommons.org/l/by-nc-sa/2.0/fr/80x15.png}}
    }
    {Distributed under the Creative Commons Attribution~--~Non
     commercial~--~Share alike 2.0 France license. See
     \url{\licenseURL} for the legal terms.}

\emph{Translation by}
Daniel C. Bünzli,
Eric Cooper,
Eliot Handelman,
Priya Hattiangdi,
Thad Meyer,
Prashanth Mundkur,
Richard Paradies,
Till Varoquaux,
Mark Wong-VanHaren

\emph{Proofread by}
David Allsopp,
Erik de Castro Lopo,
John Clements,
Anil Madhavapeddy,
Prashanth Mundkur

\emph{Translation coordination \& layout by} Daniel C. Bünzli.

\emph{英語版の日本語への翻訳: } Yuki (\href{https://github.com/inzkyk}{github})

誤訳の指摘などは \url{https://github.com/inzkyk/ocamlunix-jp/issues} まで。

% Please send corrections to \texttt{daniel.buenzl i@erratique.ch}.
\end{copyrightnotice}

\ifhtml{
% Available as a \ahref{ocamlunix.html}{monolithic file},
% \ahref{index.html}{by chapters}, and in \ahref{ocamlunix.pdf}{PDF}
% ---
% \ahref{ocamlunix-!!VERSION!!.tbz}{sources},
% git \href{http://github.com/ocaml/ocamlunix/}{repository}.}
次のフォーマットが利用可能です: \ahref{ocamlunix.html}{一つのウェブページ}, \ahref{index.html}{章ごとのウェブページ}, \ahref{ocamlunix.pdf}{PDF} --- \href{https://github.com/inzkyk/ocamlunix-jp}{git リポジトリ}}
\vfill
\begin{abstract}
% This document is an introductory course on Unix system programming,
% with an emphasis on communications between processes. The main novelty
% of this work is the use of the OCaml language, a dialect of the
% ML language, instead of the C language that is customary in systems
% programming. This gives an unusual perspective on systems programming
% and on the ML language.
この文書は Unix システムプログラミングの入門コースであり、特にプロセス間通信に重点を置いています。システムプログラミングで一般的な C 言語ではなく、ML 言語の方言である OCaml 言語を使っていることがこの文書の一番の特徴であり、これによってシステムプログラミングと ML 言語に対する普通とは異なる視点を持つことができます。
\end{abstract}
