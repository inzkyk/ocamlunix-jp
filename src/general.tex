%------------------------------------------------------------------------------
% Copyright (c) 1991-2014, Xavier Leroy and Didier Remy.
%
% All rights reserved. Distributed under a creative commons
% attribution-non-commercial-share alike 2.0 France license.
% http://creativecommons.org/licenses/by-nc-sa/2.0/fr/
%
% Translation by Eliot Handelman (eliot@colba.net)
% 英語版の日本語への翻訳: Yuki (github: inzkyk)
%------------------------------------------------------------------------------

% \chapter{Generalities}
\chapter{{基礎}}
\cutname{generalities.html}

% \section{Modules {\normalfont\texttt{Sys}} and {\normalfont\texttt{Unix}}}
\section{{\normalfont\texttt{Sys}} モジュールと {\normalfont\texttt{Unix} モジュール}}

% Functions that give access to the system from {\ocaml} are grouped into two
% modules. The first module,  \libmodule{Sys}, contains those functions
% common to Unix and other operating systems under which {\ocaml} runs.
% The second module, \libmodule{Unix}, contains everything specific to
% Unix.
\ocaml からシステムにアクセスするときに使われる関数は \ml+Sys+ と \ml+Unix+ の二つのモジュールにまとめられています。一つ目の \libmodule{Sys} モジュールは \ocaml が実行される Unixおよびその他のオペレーティングシステムで一般的な関数を含みます。二つ目の \libmodule{Unix} モジュールは Unix に特有なものが全て含まれています。

% In what follows, we will refer to identifiers from the \ml+Sys+ and
% \ml+Unix+ modules without specifying which modules they come from.  That is, we
% will suppose that we are within the scope of the directives
% \ml+open Sys+ and \ml+open Unix+. In complete examples, we explicitly write
% \ml+open+, in order to be truly complete.
これ以降 \ml+Sys+ と \ml+Unix+ モジュールにある識別子はどちらのモジュールのものかを示すことなく使うことにします。つまり \ml+open Sys+ および \ml+open Unix+ を実行した状態であるということです。完全な例を示すときには、\ml+open Sys+ と \ml+open Unix+ を明示的に書くことにします。

% The \ml+Sys+ and \ml+Unix+ modules can redefine certain
% identifiers of the \ml+Pervasives+ module, hiding previous
% definitions. For example,  \ml+Pervasives.stdin+  is different from
% \ml+Unix.stdin+. The previous definitions can always be obtained
% through a prefix.
\ml+Sys+ および \ml+Unix+ モジュールは \ml+Pervasives+ モジュールに定義されている変数を上書きし、元の定義を隠してしまうことがあるので注意してください。例えば、 \ml+Pervasives.stdin+ と \ml+Unix.stdin+ は別物です。隠された定義にはプリフィックスをつけることでアクセスできます。

% To compile an {\ocaml} program that uses the
% Unix library, do this:
Unix ライブラリを使う \ocaml のプログラムをコンパイルするには、次のようにします:
%
\begin{lstlisting}
ocamlc -o prog unix.cma mod1.ml mod2.ml mod3.ml
\end{lstlisting}
%
% where the program \ml+prog+ is assumed to comprise of the three modules \ml+mod1+,
% \ml+mod2+ and \ml+mod3+. The modules can also be compiled separately:
ここで \ml+prog+ というプログラムは \ml+mod1+, \ml+mod2+ そして \ml+mod3+ という三つのモジュールから成ります。モジュールは別々にコンパイルすることもできます:
%
\begin{lstlisting}
ocamlc -c mod1.ml
ocamlc -c mod2.ml
ocamlc -c mod3.ml
\end{lstlisting}
%
% and linked with:
この場合、次のようにしてリンクします:
%
\begin{lstlisting}
ocamlc -o prog unix.cma mod1.cmo mod2.cmo mod3.cmo
\end{lstlisting}
%
% In both cases, the argument \ml+unix.cma+ is the \ml+Unix+ library
% written in {\ocaml}. To use the native-code compiler rather than the
% bytecode compiler, replace \ml+ocamlc+ with \ml+ocamlopt+ and
% \ml+unix.cma+ with \ml+unix.cmxa+.
両方の例において、引数 \ml+unix.cma+ は \ocaml で書かれた \ml+Unix+ ライブラリを表します。バイトコードコンパイラではなくネイティブコードコンパイラを使うには、 \ml+ocamlc+ を \ml+ocamlopt+ に、\ml+unix.cma+ を \ml+unix.cmxa+ に置き換えてください。


% If the compilation tool  \ml+ocamlbuild+ is used, simply add the
% following line to the
% \ml+_tags+ file:
コンパイルツール \ml+ocamlbuild+ を使っている場合、次の内容を \ml+_tags+ ファイルに追加してください:
%
\begin{lstlisting}
<prog.{native,byte}> : use_unix
\end{lstlisting}
%
% The Unix system can also be accessed from the interactive system,
% also known as the \quotes{toplevel}. If your platform supports dynamic
% linking of C libraries, start an \ml+ocaml+ toplevel and type in the
% directive:
\quotes{toplevel} を言われる対話環境から Unix システムにアクセスすることもできます。実行している環境が C ライブラリの動的リンクに対応している場合、\ocaml トップレベルを起動して次のディレクティブを入力します:
%
\begin{lstlisting}
#load "unix.cma";;
\end{lstlisting}
%
% Otherwise, you will need to create an interactive system containing
% the pre-loaded system functions:
動的リンクに対応していない場合、システム関数がプリロードされた対話環境を作る必要があります:
%
\begin{lstlisting}
ocamlmktop -o ocamlunix unix.cma
\end{lstlisting}
%
% This toplevel can be started by:
このトップレベルは次のコマンドで起動できます:
\begin{lstlisting}
./ocamlunix
\end{lstlisting}

% \section{Interface with the calling program}
\section{プログラムを呼ぶためのインターフェース}

% When running a program from a shell (command interpreter), the shell
% passes \emph{arguments} and an \emph{environment} to the program.  The
% arguments are words on the command line that follow the name of the
% command. The environment is a set of strings of the form
% \texttt{variable=value}, representing the global bindings of environment
% variables: bindings set with \texttt{setenv var=val} for the
% \texttt{csh} shell, or with \texttt{var=val; export var} for
% the \texttt{sh} shell.
シェル (コマンドインタープリタ) からプログラムを実行する場合、シェルは \emph{引数} と \emph{環境} を実行するプログラムに渡します。引数とはコマンドライン上でプログラムの名前の後ろに続く語です。環境とは \texttt{variable=value} の形をした文字列の集まりであり、環境変数のバインディングを表します。このバインディングは csh では \texttt{setenv var=val} で、 \texttt{sh} では \texttt{var=val; export var}でセットされます。

% The arguments passed to the program are in the string array
% \ml+Sys.argv+:
プログラムに渡された引数は文字列の配列 \ml+Sys.argv+ に格納されます:
%
\begin{listingcodefile}{tmpsys.mli}
val $\indexlibvalue{Sys}{argv}$ : string array
\end{listingcodefile}
%
% The environment of the program is obtained by the function
% \ml+Unix.environment+:
プログラムの環境は \ml+Unix.environment+ 関数で取得できます:
%
\begin{listingcodefile}{tmpunix.mli}
val $\indexlibvalue{Unix}{environment}$ : unit -> string array
\end{listingcodefile}
%
% A more convenient way of looking up the environment is to use the
% function \ml+Sys.getenv+:
\ml+Sys.getenv+ 関数を使えば環境をより簡単に検索できます。
%
\begin{listingcodefile}{tmpsys.mli}
val $\indexlibvalue{Sys}{getenv}$ : string -> string
\end{listingcodefile}
%
% \ml+Sys.getenv v+ returns the value associated with the variable name
% \ml+v+ in
% the environment, raising the exception  \ml+Not_found+ if this
% variable is not bound.
\ml+Sys.getenv v+ は \ml+v+ という環境変数に結び付けられた値を返します。環境変数が見つからなかった場合は \ml+Not_found+ 例外を出します。
%
\begin{example}
% As a first example, here is the \ml+echo+ program, which prints a
% list of its arguments, as does the Unix command of the same name.
最初の例として、引数を出力する \ml+echo+ プログラムを示します。これは同じ名前の Unix コマンドと同じ動作です。
\begin{listingcodefile}{echo.ml}
let echo () =
  let len = Array.length Sys.argv in
  if len > 1 then
    begin
      print_string Sys.argv.(1);
      for i = 2 to len - 1 do
        print_char ' ';
        print_string Sys.argv.(i);
      done;
      print_newline ();
    end;;
echo ();;
\end{listingcodefile}
\end{example}

% A program can be terminated at any point with a call to \ml+exit+:
プログラムは \ml+exit+ を呼ぶことで任意の場所で終了させることができます。
%
\begin{listingcodefile}{tmppervasives.mli}
val $\indexlibvalue{Pervasives}{exit}$ : int -> 'a
\end{listingcodefile}
%
% The argument is the return code to send back to the calling program. The
% convention is to return 0 if all has gone well, and to return a
% non-zero code to signal an error. In conditional constructions, the
% \ml+sh+ shell interprets the return code 0 as the boolean
% \quotes{true}, and all non-zero codes as the boolean \quotes{false}.
引数は呼び出し元のプログラムに送られる返り値です。問題のない場合には 0 を、エラーが起こった場合には0でない値を返すという慣習があります。プログラムの実行結果が条件として使われた場合、\ml+sh+ シェルは返り値 0 をブール値 \quotes{true} に、0 でない全ての返り値を \quotes{false} として解釈します。

% When a program terminates normally after executing all of the
% expressions of which it is composed, it makes an implicit call to
% \ml+exit 0+. When a program terminates prematurely because an
% exception was raised but not caught, it makes an implicit call to
% \ml+exit 2+.
プログラムが全ての式を実行し終わって終了する場合、そのプログラムは暗黙的に \ml+exit 0+ を呼びます。プログラムが補足されない例外によって途中で実行を終了する場合、そのプログラムは暗黙的に \ml+exit 2+ を呼びます。

% The function \ml+exit+ always flushes the buffers of all channels open for
% writing. The function \ml+at_exit+ lets one register other actions
% to be carried out when the program terminates.
\ml+exit+ 関数は呼ばれたときに書き込み用にオープンされている全てのチャンネルのバッファをフラッシュします。\ml+at_exit+ 関数を使うと、プログラムが終了するときにこれ以外の動作をさせることができます。
%
\begin{listingcodefile}{tmppervasives.mli}
val $\indexlibvalue{Pervasives}{at\_exit}$ : (unit -> unit) -> unit
\end{listingcodefile}
%
% The last function to be registered is called first. A function registered with
% \ml+at_exit+ cannot be unregistered. However, this is not a
% real restriction: we can easily get the same effect with a function
% whose execution depends on a global variable.
最後に登録された関数が最初に実行されます。\ml+at_exit+ 関数を使って登録された関数は登録を解除することができませんが、これが本質的な制限になることはありません。グローバル変数を使って実行を変えることができるからです。

% \section{Error handling}
\section{エラー処理}

% Unless otherwise indicated, all functions in the \ml+Unix+ module
% raise the exception \ml+Unix_error+ in case of error.
他に明示されていない限り、\ml+Unix+ モジュールの全ての関数はエラーが起きたときに \ml+Unix_error+ 例外を出します。
%
\begin{codefile}{tmpunix.mli}
type error = Unix.error
\end{codefile}
%
\begin{listingcodefile}{tmpunix.mli}
exception $\libexn{Unix}{Unix\_error}$ of error * string * string
\end{listingcodefile}
%
% The second argument of the \ml+Unix_error+ exception is the name of
% the system call that raised the error. The third argument identifies,
% if possible, the object on which the error occurred; for example, in
% the case of a system call taking a file name as an argument, this file name will be
% in the third position in \ml+Unix_error+. Finally, the first argument
% of the exception is an error code indicating the nature of the
% error. It belongs to the variant type \ml+error+:
\ml+Unix_error+ 例外の第二引数はエラーが起こったシステムコールの名前です。第三引数はエラーが起こったオブジェクトの名前を(可能な場合には)表します。例えば、ファイルの名前を引数として取るシステムコールの場合には、このファイルの名前が \ml+Unix_error+ の第三引数となります。最後に、第一引数はエラーの種類を表すエラーコードを表します。エラーコードは \ml+error+ というヴァリアント型に属しています。
%
\begin{lstlisting}
type $\libtype{Unix}{error}$ = E2BIG | EACCES | EAGAIN | ...  | EUNKNOWNERR of int
\end{lstlisting}
%
% Constructors of this type have the same names and meanings as those
% used in the \textsc{posix} convention and  certain errors from
% \textsc{unix98} and \textsc{bsd}. All other errors use the constructor \ml+EUNKOWNERR+.
この型のコンストラクタには \textsc{posix} で定義されるエラーが同じ名前と意味ですべて含まれ、加えて \textsc{unix98}, \textsc{bsd} のエラーの一部が含まれます。その他の全てのエラーは \ml+EUNKOWNERR+ というコンストラクタになります。

% Given the semantics of exceptions, an error that is not specifically
% foreseen and intercepted by a \ml+try+ propagates up to the top of a
% program and causes it to terminate prematurely.  In small
% applications, treating unforeseen errors as fatal is a good practice.
% However, it is appropriate to display the error clearly. To do this,
% the \ml+Unix+ module supplies the \ml+handle_unix_error+ function:
例外が発生したとき、 \ml+try+ によって補足されないエラーはプログラムの一番上まで上がっていき、プログラムを実行の途中で終了させます。小さいアプリケーションでは予見できないエラーを致命的なものとみなすことは良い習慣です。しかしその場合、エラーを分かりやすく表示することが望ましいです。エラーを分かりやすく表示するために、 \ml+Unix+ モジュールには \ml+handle_unix_error+ 関数があります:
%
\begin{listingcodefile}{tmpunix.mli}
val $\indexlibvalue{Unix}{handle\_unix\_error}$ : ('a -> 'b) -> 'a -> 'b
\end{listingcodefile}
%
% The call  \ml+handle_unix_error f x+ applies function  \ml+f+ to the
% argument \ml+x+. If this raises the exception \ml+Unix_error+, a
% message is displayed describing the error, and the program is
% terminated with  \ml+exit 2+. A typical use is
\ml+handle_unix_error f x+ はまず引数 \ml+x+ を関数 \ml+f+ に適用します。この適用が \ml+Unix_error+ を出した場合、エラーを説明するメッセージが表示され、\ml+exit 2+ によってプログラムは終了します。次のプログラムは典型的な使用例です:
%
\begin{lstlisting}
handle_unix_error prog ();;
\end{lstlisting}
%
% where the function \ml+prog : unit -> unit+ executes the body of the
% program. For reference, here is how \ml+handle_unix_error+ is
% implemented.
ここで関数 \ml+prog : unit -> unit+ がプログラム本体を実行します。参考のために、 \ml+handle_unix_error+ の実装を以下に示します。
%
\begin{listingcodefile}[style=numbers]{handle_unix_error.ml}
open Unix;;
let handle_unix_error f arg =
  try
    f arg
  with Unix_error(err, fun_name, arg) ->
    prerr_string Sys.argv.(0); $\label{prog:argv}$
    prerr_string ": \"";
    prerr_string fun_name;
    prerr_string "\" failed";
    if String.length arg > 0 then begin
      prerr_string " on \"";
      prerr_string arg;
      prerr_string "\""
    end;
    prerr_string ": ";
    prerr_endline (error_message err); $\label{prog:errmsg}$
    exit 2;;
\end{listingcodefile}
%
% Functions of the form \ml+prerr_xxx+ are like the functions
% \ml+print_xxx+, except that they write on the error channel
% \ml+stderr+ rather than on the standard output channel \ml+stdout+.
\ml+prerr_xxx+ の形をした関数は基本的には \ml+print_xxx+ 関数と同じ動作をしますが、書き込み先は \ml+stdout+ ではなく \ml+stderr+ となります。

% The primitive \indexlibvalue{Unix}{error\_message}, of type
% \ml+error -> string+, returns a message describing the error given as an
% argument (line~\ref{prog:errmsg}). The argument number zero of the
% program, namely \ml+Sys.argv.(0)+, contains the name of the command
% that was used to invoke the program (line~\ref{prog:argv}).
\ml+error -> string+ 型をもつ \indexlibvalue{Unix}{error\_message} は引数の番号が表すエラーを説明するメッセージを返します (第 \ref{prog:errmsg} 行)。プログラムに渡される第 0 引数 \ml+Sys.argv.(0)+ にはプログラムを起動するのに使われたコマンドが格納されます (第 \ref{prog:argv} 行)。

% The function \ml+handle_unix_error+ handles fatal errors, \ie{} errors
% that stop the program.  An advantage of {\ocaml} is that it requires
% all errors to be handled, if only at the highest level by
% halting the program. Indeed, any error in a system call raises an
% exception, and the execution thread in progress is interrupted up to
% the level where the exception is explicitly caught and handled. This avoids
% continuing the program in an inconsistent state.
\ml+handle_unix_error+ 関数がプログラムの実行を終了させるような致命的なエラーを処理します。\ocaml を使うことの利点は全てのエラーを明示的に処理することが求められ、エラーが発生するとプログラムの実行が終了することになるトップレベルでさえこれが求められることです。実際システムコールによるどんなエラーも例外を発生させるので、プログラムの実行は中断され、例外は明示的に補足・処理されるまで上に登る事になります。これによってプログラムが不整合状態で実行が続くことを防ぐことができます。

% Errors of type \ml+Unix_error+  can, of course, be
% selectively matched. We will often see the following
% function later on:
\ml+Unix_error+ 型のエラーにはもちろんパターンマッチを使うことができます。次の関数はこれからよく目にすることになります:
%
\begin{lstlisting}
let rec restart_on_EINTR f x =
  try f x with Unix_error (EINTR, _, _) -> restart_on_EINTR f x
\end{lstlisting}
%
% which is used to execute a function and to restart it automatically
% when it executes a system call that is interrupted (see section~\ref{sec/sigsyscalls}).
このコードは関数を実行してもし中断された場合にはもう一度繰り返すという処理を行います(\ref{sec/sigsyscalls} 節を参照)。

% \section{Library functions}
\section{ライブラリ関数}

% As we will see throughout the examples, system programming often
% repeats the same patterns. To reduce the code of each application to
% its essentials, we will want to define library functions that
% factor out the common parts.
これから例を通して見ていくことですが、システムプログラミングでは同じパターンの処理が繰り返し出てきます。アプリケーションのコードが本質的な部分だけを含むように、共通する処理をまとめたライブラリ関数を定義しておくことが望ましいです。

% Whereas in a complete program one knows precisely which errors can be
% raised (and these are often fatal, resulting in the program being stopped),
% we generally do not know the execution context in the case of library functions. We
% cannot suppose that all errors are fatal. It is therefore necessary to
% let the error return to the caller, which will decide on a suitable
% course of action (\eg{} stop the program, or handle or ignore the error). However,
% the library function in general will not allow the error to simply pass
% through, since it must maintain the system in a consistent state. For
% example, a library function that opens a file and then applies an
% operation to its file descriptor must take care to close the
% descriptor in all cases, including those where the processing of the
% file causes an error. This is in order to avoid a file descriptor
% leak, leading to the exhaustion of file descriptors.
自分で書いて自分で実行するプログラムではどんなエラーが出て、そのうちどれが実行を終了させるような致命的なエラーかが分かるものですが、ライブラリ関数の場合には実行されるコンテキストが分からないのでどれが致命的なエラーなのかは通常分かりません。かといって全てのエラーが致命的であると仮定することもできません。そのためプログラムを止めるのか、それとも無視するのか呼び出し元に判断させるために、エラーを呼び出し元に伝えることが必要になります。

しかし、ライブラリ関数を普通に実装すると発生したエラーをそのまま呼び出し元に伝えることができません。システムを整合状態に置くことが求められるためです。例えばファイルを開いてそのファイルディスクリプタを使って操作を行うライブラリ関数は、ファイルへの操作でエラーが生じた場合を含めた全ての場合においてファイルディスクリプタを閉じる処理を行う必要があります。ファイルディスクリプタがリークしてファイルディスクリプタを使いきってしまうことを防ぐためです。

% Furthermore, the operation applied to a file may be defined by a
% function that was received as an argument, and we don't know precisely
% when or how it can fail (but the caller in general will know). We are
% thus often led to protect the body of the processing with
% \quotes{finalization} code, which must be executed just before the
% function returns, whether normally or exceptionally.
ファイルに対する操作を引数として受け取る場合もあります。この場合、いつどのように操作が失敗するかを知ることは (呼び出した側でなければ) できません。そのため操作の本体は \quotes{最終処理} コードで守ることが必要になります。このコードは関数が例外を出したかどうかにかかわらず、関数が帰る直前に実行されます。

% There is no built-in finalize construct \ml+try+ \ldots \ml+finalize+ in
% the {\ocaml} language, but it can be easily defined\footnote{A
% built-in construct would not be less useful.}:
\ml+try+ \ldots \ml+finalize+ 構文は \ocaml にビルトインでは用意されていませんが、簡単に定義することができます \footnote{構文がビルトインで用意されれば関数を定義して使うよりも便利でしょう。}。\begin{codefile}{misc.mli}(** miscellaneous functions for the Unix library *)

open Sys
open Unix

(** {6 Finalization} *)

val try_finalize : ('a -> 'b) -> 'a -> ('c -> unit) -> 'c -> 'b
(** [try_finalize f x g y] applies the main code [f] to [x] and
   returns the result after having applied the finalization
   code [g] to [y]. If the main code raises the exception
   [exn], the finalization code is executed and [exn] is raised.
   If the finalization code itself fails, the exception
   returned is always the one from the finalization code. *)
\end{codefile}
%
\begin{listingcodefile}{misc.ml}
let try_finalize f x finally y =
  let res = try f x with exn -> finally y; raise exn in
  finally y;
  res
\end{listingcodefile}
%
% This function takes the main body \ml+f+ and the finalizer
% \ml+finally+, each in the form of a function, and two parameters \ml+x+
% and \ml+y+, which are passed to their respective functions. The body
% of the program \ml+f x+ is executed first, and its result is kept
% aside to be returned after the execution of the finalizer
% \ml+finally+. In case the program fails, \ie{} raises an exception \ml+exn+,
% the finalizer is run and the exception \ml+exn+ is raised
% again. If both the main function and the finalizer fail, the
% finalizer's exception is raised (one could choose to have the main
% function's exception raised instead).
この関数はメインの処理 \ml+f+ と最終処理 \ml+finally+ およびそれらの引数 \ml+x+ と \ml+y+ を受け取ります。最初にプログラムの本体 \ml+f x+ が実行され、その結果は最終処理 \ml+finally y+ が実行されてから返されます。プログラムの実行で例外 \ml+exn+ が起こった場合、最終処理が実行されてからもう一度 \ml+exn+ を出します。メインの処理と最終処理の両方が失敗した場合、最終処理の例外が出されます(メインの処理の例外が出されるようにすることもできます)。

\paragraph{ノート}

% In the rest of this course, we use an auxiliary library \ml+Misc+
% which contains several useful functions like \ml+try_finalize+ that are often
% used in the examples. We will introduce them as they are needed. To
% compile the examples of the course, the definitions of the \ml+Misc+
% module need to be collected and compiled.
これからこのコースでは例でよく使う \ml+try_finalize+ などの関数をまとめた補助ライブラリ \ml+Misc+ を使います。必要に応じてライブラリの関数を紹介するほか、インターフェースは付録にあります。このコースに出てくる例をコンパイルするには、 \ml+Misc+ モジュールの定義をまとめておく必要があります。

% The \ml+Misc+ module also contains certain functions, added for
% illustration purposes, that will not be used in the course. These
% simply enrich the \ml+Unix+ library, sometimes by redefining the
% behavior of certain functions.  The \ml+Misc+ module must thus take
% precedence over the \ml+Unix+ module.
\ml+Misc+ モジュールにはこのコースでは直接使用しない可視化のための関数も含まれています。これらの関数は \ml+Unix+ ライブラリを強化するためのもので、いくつかの関数の振る舞いを上書きします。そのため \ml+Misc+ ライブラリを使う場合は \ml+Unix+ の後に読み込まれる必要があります。

\paragraph{例}

% The course provides numerous examples. They can be compiled with
% {\ocaml}, version {\ocamlversion}$\!\!\!$.  %% To get rid of that space !
% Some programs will have to be slightly modified in order to work with
% older versions.
このコースにはたくさんの例が含まれています。 これらはバージョン \ocamlversion の \ocamlでコンパイルされることを確認しています。古いバージョンではプログラムを若干改変する必要があります。

% There are two kinds of examples: \quotes{library functions} (very
% general functions that can be reused) and small applications. It is
% important to distinguish between the two. In the case of library functions, we
% want their context of use to be as general as possible. We will thus
% carefully specify their interface and attentively treat all
% particular cases. In the case of small applications, an error is often
% fatal and causes the program to stop executing. It is sufficient to report
% the cause of an error, without needing to return to a consistent state, since
% the program is stopped immediately thereafter.
例には二つの種類があります: とても一般的で再利用が可能な \quotes{ライブラリ関数} と小さなアプリケーションです。これら二つを区別することは重要です。ライブラリ関数の場合には実行時のコンテキストをできるだけ一般的なものと仮定して、インターフェースを熟慮し、全ての特殊ケースを扱うようにします。一方小さなアプリケーションの場合には、多くのエラーは致命的なものであり、プログラムの実行を停止させます。そのためエラーが起きた時にはその原因を伝えるだけで十分であり、システムを整合状態へと戻す処理は必要ありません。
